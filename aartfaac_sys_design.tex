\documentclass{aa}
\usepackage[varg]{txfonts}
\usepackage{color}
\usepackage{graphicx}
\usepackage{subfig}    %% For generating a grid of figures.
%% \newsavebox\mybox   %% For generating figure captions which wrap at figure 
%% edges.
%% \newlength\myboxlen
%% \newcommand{\figcap}[2]
%% {%
%%   \sbox\mybox{#1}
%%   \settowidth{\myboxlen}{\usebox{\mybox}}
%%   \centering
%%   \usebox\mybox
%%   \hskip \textwidth
%%   \parbox{\myboxlen}{#2}
%% }

\usepackage{bm}
\bibpunct{(}{)}{;}{a}{}{,} % to follow the A&A style
\begin{document}

\title{Design and Commissioning of the AARTFAAC all-sky monitor}

\author{Peeyush Prasad  \inst{1,2} \and Folkert Huizinga \inst{1} \and John Romein \inst{2}
\and Daniel van der Schuur \inst{2} \and Ralph  Wijers  \inst{1}}
 \institute{Universiteit  van
  Amsterdam \and ASTRON, The Netherlands Foundation for Radio Astronomy}

\date{Received <date> / Accepted <date>}

%% Structured abstract
\abstract{The  Amsterdam-ASTRON Radio  Transients Facility  And Analysis  Center
  (AARTFAAC)  array is  a  sensitive,  all-sky radio  imager  based  on the  Low
  Frequency Array  (LOFAR). It will  provide images  in near real-time  with sub
  arcmin resolutions at low radio frequencies, which will be monitored for short
  and bright radio transients. On detection of a transient, low latency triggers
  will be generated for LOFAR, which can carry out follow-up observations.
  In this paper, we describe the  implementation of the instrumentation, and its
  capabilities.}

  \keywords{Radio Interferometry - Imaging - Radio Transients - Correlators}

\maketitle

\section{\label{sec:Introduction}Introduction}
The AARTFAAC radio  transient monitor is a  leading effort among a  group of new
radio  telescopes with  similar functionalities  as the  wide field  monitors at
higher  energies.  They are  characterized  by  having moderate  resolution  and
sensitivity as compared  to contemporary telescopes, with  extremely wide fields
of view (typically all sky),  high availabilities and autonomous calibration and
imaging.  The  latter requirements  make their implementations  challenging. The
antenna elements used to achieve the  wide fields of view are typically dipoles,
however,  their low  individual  sensitivities requires  an  order of  magnitude
larger number of  elements in the array. Bringing the  resulting large number of
data streams to  a central location, as  well as their correlation  thus poses a
significant  I/O  and  compute  challenge.   The wide  fields  of  view  at  the
sensitivities of  operation also  result in direction  dependent effects  on the
incoming  signals, mostly  due  to the  ionosphere. These  pose  a challenge  to
calibration, especially when carried out in an autonomous manner.

In this  paper, we describe  the implementation  of the instrumentation  for the
AARTFAAC  array,  and the  commissioning  of  its various  subsystems.   Section
\ref{sec:aartfaac_array} describes the array and the receiving antenna elements,
its  relationship  with LOFAR,  and  introduces  the  full architecture  of  the
instrument.    Section   \ref{sec:station_hardware}   describes   the   hardware
implementation in  the field which  allows creating a  data path in  parallel to
LOFAR. This  makes AARTFAAC processing independent  of LOFAR to a  large extent.
In Section \ref{sec:gpucorr}, we describe the implementation of a real-time, GPU
based  correlator  for  AARTFAAC,  while  Section  \ref{sec:calim}  details  the
real-time,   autonomous  calibration   and   imaging  implementation.    Section
\ref{sec:acontrol} describes our  control system for the  full instrument, which
also interfaces with LOFAR.  In Section \ref{sec:results} we present performance
metrics of the instrument as a whole.

\section {\label{sec:aartfaac_array}The AARTFAAC array}
\begin {itemize}
 \item {Requirements for transient detection}
 \item {Station description, Analog bandwidth, antenna field of view}
 \item {RSP signal processing, available bit modes}
 \item {AARTFAAC-12 array configuration}
 \item {Effect of regular lofar observations on AARTFAAC, fraction of time spent
   by LOFAR in HBA, LBA\_OUTER, LBA\_INNER mode}
 \item {AARTFAAC system specs.}
\end {itemize}

\section {\label{sec:station_hardware} Station level hardware for piggy-back operation}
\begin {itemize}
 \item {URI board  description, data coupling scheme,  constraints on achievable
   bandwidth}
 \item  {The uniboard  data  reformatting (transpose),  uniboard data  transfer,
   output data format}
 \item {Available diagnostics, performance, commissioning tests}
\end{itemize}

\section {\label{sec:gpucorr} The AARTFAAC real-time correlator}
\begin {itemize}
 \item {Correlation for transit mode observations: logical blocks.}
 \item {Description of processing flow.} 
 \item {Motivation of chosen architecture for implementation.}
 \item {Supported time and frequency binning, motivation of choice.}
 \item {Required compute and memory bandwidth.}
 \item {Synchronization of incoming data (input buffer), output data format.}
 \item {Commissioning tests, performance.}
\end {itemize}

\section {\label{sec:calim} Real-time calibration and imaging}
\begin {itemize}
 \item {Architecture, implementation choices, performance}
 \item {Unit test architecture}
 \item {Interface to TraP}
\end {itemize}

\section {\label{sec:acontrol} The AARTFAAC control interface}
\begin {itemize}
 \item {Control system description}
 \item {Interface with LOFAR}
 \item {Monitoring interface: AARTFAAC webpage}
\end {itemize}

\section {\label{sec:results} Results}
\begin {itemize}
 \item {Long term performance of the entire system based on logs.}
 \item {Performance  in various  bit-modes, with  different number  of subbands,
   expected sensitivity}
\end {itemize}

\section {\label{sec:discussion} Discussion}

\section {\label{sec:conclusion} Conclusions}

\begin {acknowledgements}

This work  was funded  by the ERC  grant <num> awarded  to Prof.   Ralph Wijers,
Universitiet  Van Amsterdam.   We  thank The  Netherlands  Foundation for  Radio
Astronomy  (ASTRON)  for support  provided  in  carrying out  the  commissioning
observations.
\end{acknowledgements}
\bibliographystyle{aa}
\bibliography{ref}

\end{document}
